% Created 2013-02-25 Mon 04:21
\documentclass[11pt]{article}
\usepackage{listings}
\usepackage[utf8]{inputenc}
\usepackage[T1]{fontenc}
\usepackage{fixltx2e}
\usepackage{graphicx}
\usepackage{longtable}
\usepackage{float}
\usepackage{wrapfig}
\usepackage{soul}
\usepackage{textcomp}
\usepackage{marvosym}
\usepackage{wasysym}
\usepackage{latexsym}
\usepackage{amssymb}
\usepackage{hyperref}
\tolerance=1000
\author{Grégoire Jadi}
\date{\today}
\title{}
\hypersetup{
  pdfkeywords={},
  pdfsubject={},
  pdfcreator={Generated by Org mode 7.9.3e in Emacs 24.3.50.2.}}
\begin{document}

\tableofcontents

\section[Les premières idées]{Les premières idées}
\label{sec-1}
\begin{itemize}
\item Note taken on \textit{[2013-02-21 Thu 10:40]} \\
    Les premiers dessins + mini scénarios de tests
\item Note taken on \textit{[2013-02-21 Thu 11:12]} \\
    Tableau +/- à la fin de chaque présentations d'interfaces
\end{itemize}

\subsection[Une interface sobre]{Une interface sobre}
\label{sec-1-1}

\begin{itemize}
\item on voit beaucoup de choses à la fois
\item interface "de travail"
\end{itemize}

⇒ Techno centré

\subsection[Une interface connue]{Une interface connue}
\label{sec-1-2}
\begin{itemize}
\item vue comme l'arborescence de fichiers
→ utilisation métaphore connue
\end{itemize}
\subsection[Une interface sexy]{Une interface sexy}
\label{sec-1-3}

\begin{itemize}
\item inspirée de Org Mode
\item ergonomique
\item user centrée
\end{itemize}
\subsection[Bilan et comparaison]{Bilan et comparaison}
\label{sec-1-4}
\section[Un premier design]{Un premier design}
\label{sec-2}
\begin{itemize}
\item Note taken on \textit{[2013-02-21 Thu 10:48]} \\
    scénarios + storyboard
\end{itemize}

\subsection[Le principe d'accordéon]{Le principe d'accordéon}
\label{sec-2-1}
\begin{itemize}
\item cacher/afficher
\item intuitif
\item smartphone lire
\item absence dans Qt de l'animation
\end{itemize}
\subsection[Des boutons + en pagaille]{Des boutons + en pagaille}
\label{sec-2-2}
\begin{itemize}
\item différencier la tâche de la sous-tâche
\item "one click action"
\end{itemize}
\subsection[No pop paradigm]{No pop paradigm}
\label{sec-2-3}
\begin{itemize}
\item Modifier une tâche à la volée
\item Pas d'avertissement à la suppression (undo/redo de prévu)
\end{itemize}
\subsection[Les fonctionnalités]{Les fonctionnalités}
\label{sec-2-4}
\begin{itemize}
\item ajout, expand, order, close, update, check, link date, up\&down
\item le "all one click"
\item template via click droit
\end{itemize}
\section[Arrivée d'un expert]{Arrivée d'un expert}
\label{sec-3}

\subsection[Trop de +]{Trop de +}
\label{sec-3-1}
\begin{itemize}
\item ambiguïté
\item pas instinctif
\item passage à l'echelle (plein de tâches, plein de boutons, plein de
boutons, plein de boutons\ldots{})
\end{itemize}
\subsection[« Et le bouton paramètre ? »]{« Et le bouton paramètre ? »}
\label{sec-3-2}
\begin{itemize}
\item Résumer les actions possibles → Everything in one button
limitation de la taille en changeant le texte (order/unorder)
\end{itemize}
\subsection[Ordonnancement]{Ordonnancement}
\label{sec-3-3}
\begin{itemize}
\item pas instinctif
\item pas vraiment une action majeure
→ suppressoin du "one click"
\end{itemize}
\subsection[Les actions majeures]{Les actions majeures}
\label{sec-3-4}
\begin{itemize}
\item check, expand, edit, close
→ one click
\end{itemize}
\subsection[Les actions mineures]{Les actions mineures}
\label{sec-3-5}
\begin{itemize}
\item add, templates, order, up\&down, sélection date
→ two clicks
\end{itemize}
\section[Placements]{Placements}
\label{sec-4}
\begin{itemize}
\item Note taken on \textit{[2013-02-21 Thu 10:54]} \\
    paper prototyping
\end{itemize}

\subsection[Des groupes]{Des groupes}
\label{sec-4-1}
\begin{itemize}
\item ajout, visualisation, résumé actions
\item description → car impartant
\item suppression → car critique
\end{itemize}
\subsection[Le bouton paramètre]{Le bouton paramètre}
\label{sec-4-2}
\begin{itemize}
\item votes
\end{itemize}
\subsection[La croix]{La croix}
\label{sec-4-3}
\begin{itemize}
\item votes
\end{itemize}
\subsection[Ajout d'une tâches principale]{Ajout d'une tâches principale}
\label{sec-4-4}
\begin{itemize}
\item arrivée sur le logiciel → interface vide → un seul bouton
\end{itemize}
\section[Les couleurs]{Les couleurs}
\label{sec-5}

\subsection[Les tâches]{Les tâches}
\label{sec-5-1}
\begin{itemize}
\item faite, à faire, en retard
\end{itemize}
\subsection[L'édition]{L'édition}
\label{sec-5-2}
\begin{itemize}
\item le hover
\end{itemize}
\subsection[La croix]{La croix}
\label{sec-5-3}
\section[Le vocabulaire]{Le vocabulaire}
\label{sec-6}

\begin{itemize}
\item ajouter tâche vs nouvelle tâche
\item tâche vs sous-tâche
\item template
\end{itemize}
\section[Les langues]{Les langues}
\label{sec-7}
\begin{itemize}
\item menu édition
\item ajout d'une nouvelle langue rapide et facile
\end{itemize}
\section[Future works]{Future works}
\label{sec-8}
\begin{enumerate}
\item Drag\&drop
\item Shortcuts
\item Aide
\item Tout plier/déplier
\item Paramétrage des couleurs
\item Identité du logiciel (logo)
\item griser/supprimer le bouton expand
\end{enumerate}
\section[Bilan \& Conclusion]{Bilan \& Conclusion}
\label{sec-9}
\begin{itemize}
\item avantages expérimentations papier
\item prises en comptes retours utilisateurs
\end{itemize}
% Generated by Org mode 7.9.3e in Emacs 24.3.50.2.
\end{document}
