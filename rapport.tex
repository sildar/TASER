\documentclass[11pt]{article}

\usepackage{listings}
\usepackage[utf8]{inputenc}
\usepackage[T1]{fontenc}
\usepackage[french]{babel}
\usepackage{fixltx2e}
\usepackage{graphicx}
\usepackage{longtable}
\usepackage{float}
\usepackage{wrapfig}
\usepackage{soul}
\usepackage{textcomp}
\usepackage{marvosym}
\usepackage{wasysym}
\usepackage{latexsym}
\usepackage{amssymb}
\usepackage{hyperref}

\usepackage{glossaries}

%%glossary entries

\newglossaryentry{tachePrincipale}
{
  name=tâche principale,
  description={Une tâche n'ayant aucune tâche mère}
}

\newglossaryentry{soustache}
{
  name=sous-tâche,
  description={Une tâche liée à une tâche mère. Une relation de
    dépendance existe. Notamment, si une tâche mère est supprimée,
    toutes ses tâches filles (ses sous-tâches), sont supprimées}
}

\newglossaryentry{technocentre}
{
  name=technocentré,
  description={Se dit d'une interface basée sur la technologie. On se
  sert des capacités connues d'une machine pour construire une interface à
  l'image de la machine}
}

\newglossaryentry{anthropocentre}
{
  name=anthropocentré,
  description={Se dit d'une interface basée sur l'humain. On se sert
    des repères culturels pour construire une interface à l'image de l'homme}
}

\newglossaryentry{metaphore}
{
  name=métaphore,
  description={Terme utilisé pour décrire un type d'interface
    particulier. La métaphore de l'arbre est une interface proposant
    une vue de type arbre ie chaque élement peut être parent d'un
    autre élément du même type. Ici, une tâche peut être mère d'une
    autre tâche. La seconde dépend alors de la première}
}

\makeglossaries

\author{Grégoire Jadi \and{} Rémi Bois \and{} Loïc Jankowiak}
\title{TASER \\
Gestionnaire Avancé de Tâches}

\begin{document}

\maketitle
\tableofcontents


\section{Présentation}

L'objectif de ce projet était de mettre en pratique les différentes
techniques permettant de réaliser des interfaces graphiques.

À l'aide du framework graphique Qt, le but était d'imaginer une
interface homme machine (IHM) qui permettant la gestion avancée de
tâches. L'utilisateur devait entre autre pouvoir
\begin{itemize}
\item ordonner des tâches;
\item fixer une date aux tâches;
\item gérer des templates de tâches;
\end{itemize}
On se référera au sujet pour une liste exhaustive des fonctionalités
attendues.

Nous allons maintenant présenter notre implémentation,
TASER\footnote{TASk managER}, détailler et expliquer les différents
choix d'interface et d'ergonomie qui ont été fait.


\section{Les premières idées}

Une fois que nous avons pris connaissance du sujet, nous nous sommes
mis d'accord pour chacun réfléchir de notre côté pendant une semaine
sur le sujet et ainsi nous retrouver avec chacun une proposition
d'interface non biaisée.

La confrontation des idées fut très intéressante car, comme nous
allons le voir, nous avons tous présenté une interface différente qui
présentent toutes des caractéristiques distinctes.

% TODO rajouter les premiers dessins + mini scénario de chaque interface
% TODO rajouter un tableau +/- pour chaque interface


\subsection{Une interface sobre}

% TODO insérer un dessin

Comme on peut le constater, la première interface est relativement sobre.

L'utilisation de deux cadres est intéressante mais c'est une
organisation de l'espace de travail qui amènerait à montrer trop
d'informations à la fois.

De plus, c'est une interface qui rappelle beaucoup les grosses
applications dédiées à un domaine, elle est très « \gls{technocentre} ».


\subsection{Une interface connue}

% TODO insérer un dessin

Cette deuxième interface utilise donc une vue arborescente des tâches
qui possède l'avantage d'être connue. En effet, la \gls{metaphore} de
l'arbre est déjà largement utilisée par les explorateurs de fichiers.
C'est pourquoi la réutilisation de ce design permet à l'utilisateur
d'être en terrain connu.

Cependant, bien que cette disposition soit plus légère que celle de
l'interface précédente, l'abre reste assez lourd visuellement.


\subsection{Une interface sexy}

% TODO insérer un dessin

La dernière interface est fortement inspirée d'Org Mode, un mode
d'Emacs\footnote{Emacs est un excellent système d'exploitation à qui
  il ne manque qu'un éditeur de texte.} qui permet entre autre de
gérer des tâches.

Cette configuration est assez proche de la précédente, car elle
conserve l'idée d'organiser les tâches sous forme d'abres. Elle nous a
semblé être celle se rapprochant le plus d'une idée intuitive la
représentation de tâches et de \gls{soustache}. Nous l'avons considérée \gls{anthropocentre}e.

De plus, comme l'un d'entre nous utilise beaucoup Org Mode, nous
savions que c'était une interface qui était utilisable.


\subsection{Bilan}

Il nous semble qu'avoir initialement réfléchi séparément fut une bonne
idée. Cela a permis à chacun de se faire sa propre idée de l'interface
qu'il désirait et d'avoir beaucoup d'idées différentes dès le départ.

De plus, comme nous avons fait le choix de la troisième interface assez
rapidement, nous étions relativement confiant dans notre choix,
puisque nous en avions déjà étudié deux autres.

Durant cette première étape nous avons donc décidé de la structure
générale de l'interface que nous souhaitions développer.

% TODO lien en annexes pour les storyboard/images restantes

\section{Un premier design}
\label{sec:premierDesign}

Une fois que nous avions l'idée globale de l'interface désirée. Nous
avons commencé à analyser plus en détail les différentes fonctions et
modes de fonctionnement de notre interface.


\subsection{Le principe d'accordéon}

Nous souhaitions que notre interface puisse permettre l'affichage de
nombreuses tâches, sans pour autant perdre l'utilisateur dans trop de
données. Il nous fallait donc trouver une solution qui autoriserait
l'utilisateur à afficher uniquement ce qu'il décide voir, tout en
laissant l'apparition de données supplémentaires à la portée d'un
seul click. Le paradigme de l'accordéon, classique dans l'univers des
smartphones, nous a semblé être la solution adaptée.

Le paradigme de l'accordéon consiste à plier l'affichage : une tâche
contient toutes ses sous tâches, et si l'on déplie une tâche, ses sous
tâches s'affichent. Cet affichage est instinctif pour les utilisateurs
de smartphones et/ou de tablettes, friands de ce paradigme.

Bien que fréquent sur les appareils mobiles, Qt ne propose pas de
Widget adoptant ce comportement. Nous nous sommes d'abord assurés que
simuler un tel comportement était possible via l'utilisation de layers
et de Widgets classiques. Ceci a été une réussite à l'exception du
fait qu'il nous a été impossible d'utiliser une
animation. L'indentation des sous-tâches par rapport leur tâche mère
nous a semblée suffisante pour que l'utilisateur repère aisément
quelles tâches s'étaient dépliées, et nous avons donc choisi de
remettre l'utilisation d'une éventuelle animation à une mise à jour
future. 


% \begin{itemize}
% \item cacher/afficher
% \item intuitif
% \item smartphone lire
% \item absence dans Qt de l'animation
% \end{itemize}


\subsection{Des boutons + en pagaille}

L'un des principaux challenges présentés par notre interface était de
permettre l'ajout de tâches ou de sous tâches de façon aisée. Nous
avons pour cela choisi de proposer un bouton + par choix
possible. Cela encombre l'écran mais nous estimons alors que
l'indentation permet une différentiation suffisante pour connaître la
fonction de chacun des boutons + (un bouton indenté créera une sous
tâche, un bouton au même niveau qu'une tâche créera une tâche soeur).

Nous avons fait ce choix afin de pouvoir ajouter en un clic une tâche
ou une sous-tâche.


% \begin{itemize}
% \item différencier la tâche de la sous-tâche
% \item "one click action"
% \end{itemize}


\subsection{No pop-up paradigm}

Nous avons décidé de ne faire surgir aucun pop-up tout au long de
l'utilisation de notre application. D'une part, les pop-up sont très
désagréables pour les personnes souffrant de problèmes visuels,
d'autre part, nous avons estimé que l'interface serait suffisamment
claire et instinctive pour ne nécessiter aucune ouverture de fenêtre
supplémentaire.

L'une des premières conséquences de ce choix est l'obligation de
proposer la modification d'une tâche ``à la volée'' et non une
interface particulière à la modification de tâche. Nous avons donc
choisi d'implémenter un Widget dont le texte est éditable après un
clic sur celui-ci.

Le principal inconvénient de ce choix est le non avertissement lors
d'une action irréversible. C'est notamment le cas pour la suppression
d'une tâche et de ses sous-tâches. Cet inconvénient est contrebalancé
par la volonté d'alors d'implémenter un annuler/rétablir (undo/redo)
qui permettrait le retour en arrière en cas d'erreur.

% \begin{itemize}
% \item Modifier une tâche à la volée
% \item Pas d'avertissement à la suppression (undo/redo de prévu)
% \end{itemize}


\subsection{Les fonctionnalités}
\begin{itemize}
\item ajout, expand, order, close, update, check, link date, up\&down
\item le "all one click"
\item template via click droit
\end{itemize}

% TODO lien en annexes pour les storyboard/images/scénarios restants


\section{Arrivée d'un expert}

Après avoir fixé l'interface présentée en section
\ref{sec:premierDesign} (page \pageref{sec:premierDesign}) et s'être
assurés de la faisabilité technique de notre projet, nous avons décidé
de proposer à une personne indépendante de nous donner son avis sur
l'interface. Nous avons procédé en trois étapes :

\begin{itemize}
\item Présentation du principe du logiciel à l'expert (présentation
  purement orale de l'entête du sujet qui nous était proposé)
\item Questionnements sur le paper prototype (ex : ``Comment ajouter
  une tâche ?'', ``A quoi peut servir ce bouton selon vous ?'', \dots)
\item Questionnements sur le design général, les éventuels manques,
  l'impression globale, \dots
\end{itemize}

Cette section présente les retours reçus lors de cet entretient et les
choix qui en ont découlés ainsi que les raisons de ces
changements. Dans tous les cas ayant porté à débat, un vote a eu lieu,
avec voix majoritaire à l'expert, partant du principe que le logiciel
serait conçu pour un utilisateur lambda, et pas pour des étudiants en
informatique, forcément plus au fait des atouts techniques, souvent
incompatibles avec des atouts d'ergonomie.

On peut trouver en annexe un tableau descriptif du profil de l'expert
(section \ref{sec:profilExpert} page \pageref{sec:profilExpert}).

\subsection{Trop de +}

La première et majeure remarque de l'expert a été la présence trop
importante de boutons +. Bien qu'il nous semblait que l'indentation
suffisait à repérer la fonctionnalité proposée par chacun de ces
boutons, l'expert a quant à lui semblé perdu sur les différences entre
ces boutons.

L'expert a également fait remarqué qu'avec ce système, on pourrait se
retrouver dans certains cas avec plus de boutons + que de tâches
affichées, masquant l'information primordiale. Un exemple de ce genre
de situation est donné en annexe page \pageref{ann:plusplusplus}.

Nous avons donc décidé de revoir le système d'ajout de tâches et de
sous tâches. Nous avons pour cela utilisé une autre remarque de
l'expert, décrit à la section \ref{subsec:parameterButton} page \pageref{subsec:parameterButton}.


% \begin{itemize}
% \item ambiguïté
% \item pas instinctif
% \item passage à l'echelle (plein de tâches, plein de boutons, plein de
%   boutons, plein de boutons\ldots{})
% \end{itemize}


\subsection{« Et le bouton paramètre ? »}
\label{subsec:parameterButton}
\begin{itemize}
\item Résumer les actions possibles → Everything in one button
  limitation de la taille en changeant le texte (order/unorder)
\end{itemize}


\subsection{Ordonnancement}

L'ordonnancement des tâches nous semblait un point important, et donc
une action qui devait être atteignable très rapidement. L'expert
a néanmoins émis l'avis qu'il s'agissait d'une fonctionnalité
secondaire. De plus, notre idée de bouton cliquable pour passer de
tâches ordonnées à tâches non ordonnées n'était absolument pas
instinctif selon l'expert. Nous avions choisi d'afficher un simple
tiret - lorsque les tâches n'étaient pas ordonnées. En cliquant sur ce
tiret, les tâches s'ordonnaient et un chiffre apparaissait alors en
lieu et place du tiret. Peut être que le passage ordonné vers non
ordonné aurait été plus instinctif (on reconnait plus rapidement
l'usage d'un chiffre que l'usage du tiret). Néanmoins, nous avons
décidé de suivre l'avis de l'expert et de retirer l'ordonnancement des
fonctionnalités accessibles en un seul clic. Nous l'avons donc rendu
disponible uniquement via le bouton paramètre, décrit en section
\ref{subsec:parameterButton} page \pageref{subsec:parameterButton},
dont les descriptions textuelles permettaient de désambiguiser la
symbolique attribuée au tiret. 

% \begin{itemize}
% \item pas instinctif
% \item pas vraiment une action majeure
%   → suppressoin du "one click"
% \end{itemize}


\subsection{Les actions majeures}
\begin{itemize}
\item check, expand, edit, close
  → one click
\end{itemize}


\subsection{Les actions mineures}
\begin{itemize}
\item add, templates, order, up\&down, sélection date
  → two clicks
\end{itemize}


\section{Placements}
\begin{itemize}
\item Note taken on \textit{[2013-02-21 Thu 10:54]} \\
  paper prototyping
\end{itemize}


\subsection{Des groupes}
\begin{itemize}
\item ajout, visualisation, résumé actions
\item description → car impartant
\item suppression → car critique
\end{itemize}


\subsection{Le bouton paramètre}
\begin{itemize}
\item votes
\end{itemize}


\subsection{La croix}
\label{subsec:croixPlacement}
\begin{itemize}
\item votes
\end{itemize}


\subsection{Ajout d'une tâche principale}

L'une des fonctionnalités critiques à mettre en avant était l'ajout
d'une \gls{tachePrincipale}. Nous souhaitions cette fonctionnalité non
ambigüe. Lors de sa première utilisation de notre logiciel,
l'utilisateur n'aurait comme seule action possible l'ajout d'une
nouvelle tâche principale, c'est pourquoi nous voulions un bouton
toujours visible et inéquivoque.

Nous avons donc choisi de placer ce bouton tout en haut à gauche de
l'interface. Il reste toujours visible, permettant d'ajouter une
nouvelle tâche principale à n'importe quel moment. C'est le seul
bouton visible au premier lancement de l'application.


\section{Les couleurs}

L'utilisation de couleurs pertinentes est primordiale dans la
conception d'interfaces pratiques et agréables à utiliser. Nous
aurions souhaité proposer un paramétrage des couleurs utilisées (voir
section \ref{sec:futureWorks} page \pageref{sec:futureWorks}) afin de
prendre en compte les différentes cultures ainsi que les éventuels
handicaps visuels ou préférences de l'utilisateur.

Après avoir réalisé que nous n'aurions pas le temps de proposer cette
fonctionnalité de choix de couleurs, nous avons opté pour des couleurs
que nous avons jugées agréable à l'oeil, et correspondant à notre culture.


\subsection{Les tâches}
\begin{itemize}
\item faite, à faire, en retard
\end{itemize}


\subsection{L'édition}
\begin{itemize}
\item le hover
\end{itemize}


\subsection{La croix}

Nous avons décidé, en plus de la séparation décrite en section
\ref{subsec:croixPlacement} page \pageref{subsec:croixPlacement} de
signaler la suppression de tâches par une couleur rouge, sinonyme de
danger dans notre culture. Ceci est fait pour mettre en emphase le
fait qu'il s'agit d'une action irréversible, et donc qu'il ne faudrait
pas cliquer sans être sûr de ce que l'on souhaite.


\section{Le vocabulaire}

\begin{itemize}
\item ajouter tâche vs nouvelle tâche
\item tâche vs sous-tâche
\item template
\end{itemize}


\section{Les langues}

Le menu édition comporte les paramètres éditables de notre
application. L'un de ces paramètre est la langue utilisée. Un simple
click sur une autre langue que la langue courante modifie
l'intégralité des textes des boutons et menus.

Nous avons utilisé les capacités de Qt pour permettre l'ajout très
simple de nouvelles langues. Une précondition pour cela a été
l'utilisation de l'encoding UTF-8 pour les textes des boutons et
menus.

Il suffit de quelques minutes à une personne bilingue pour proposer
une nouvelle langue, via les interfaces de Qt Linguist.


\section{Future works}
\label{sec:futureWorks}
\begin{enumerate}
\item Drag\&drop
\item Shortcuts
\item Aide
\item Tout plier/déplier
\item Paramétrage des couleurs
\item Identité du logiciel (logo)
\item griser/supprimer le bouton expand
\end{enumerate}


\section{Bilan \& Conclusion}
\begin{itemize}
\item avantages expérimentations papier
\item prises en comptes retours utilisateurs
\end{itemize}

\addcontentsline{toc}{section}{Glossary}
\printglossaries

\appendix

\section{Profil de l'expert}
\label{sec:profilExpert}

\begin{tabular}[h]{c|c}
  Niveau d'études & bac + 3\\
  Fréquence d'utilisation d'un PC & quelques heures par jour\\
  Fréquence d'utilisation d'un smartphone & plusieurs heures par jour\\
  Connaissances en programmation & aucune\\
  Utilisation d'un gestionnaire de tâches & non\\

\end{tabular}

\section{Working copies}

\subsection{Trop de boutons +}
\label{ann:plusplusplus}



\end{document}
