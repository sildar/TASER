\documentclass[11pt]{article}

\usepackage{listings}
\usepackage[utf8]{inputenc}
\usepackage[T1]{fontenc}
\usepackage{fixltx2e}
\usepackage{graphicx}
\usepackage{longtable}
\usepackage{float}
\usepackage{wrapfig}
\usepackage{soul}
\usepackage{textcomp}
\usepackage{marvosym}
\usepackage{wasysym}
\usepackage{latexsym}
\usepackage{amssymb}
\usepackage{hyperref}

\author{Grégoire Jadi \and{} Rémi Bois \and{} Loïc Jankowiak}
\title{TASER \\
Gestionnaire Avaancé de Tâches}

\begin{document}

\maketitle
\tableofcontents


\section{Les premières idées}
\begin{itemize}
\item Note taken on \textit{[2013-02-21 Thu 10:40]} \\
  Les premiers dessins + mini scénarios de tests
\item Note taken on \textit{[2013-02-21 Thu 11:12]} \\
  Tableau +/- à la fin de chaque présentations d'interfaces
\end{itemize}


\subsection{Une interface sobre}
\begin{itemize}
\item on voit beaucoup de choses à la fois
\item interface "de travail"
\end{itemize}

⇒ Techno centré


\subsection{Une interface connue}
\begin{itemize}
\item vue comme l'arborescence de fichiers
  → utilisation métaphore connue
\end{itemize}


\subsection{Une interface sexy}
\begin{itemize}
\item inspirée de Org Mode
\item ergonomique
\item user centrée
\end{itemize}


\subsection{Bilan et comparaison}


\section{Un premier design}
\begin{itemize}
\item Note taken on \textit{[2013-02-21 Thu 10:48]} \\
  scénarios + storyboard
\end{itemize}


\subsection{Le principe d'accordéon}
\begin{itemize}
\item cacher/afficher
\item intuitif
\item smartphone lire
\item absence dans Qt de l'animation
\end{itemize}


\subsection{Des boutons + en pagaille}
\begin{itemize}
\item différencier la tâche de la sous-tâche
\item "one click action"
\end{itemize}


\subsection{No pop paradigm}
\begin{itemize}
\item Modifier une tâche à la volée
\item Pas d'avertissement à la suppression (undo/redo de prévu)
\end{itemize}


\subsection{Les fonctionnalités}
\begin{itemize}
\item ajout, expand, order, close, update, check, link date, up\&down
\item le "all one click"
\item template via click droit
\end{itemize}


\section{Arrivée d'un expert}


\subsection{Trop de +}
\begin{itemize}
\item ambiguïté
\item pas instinctif
\item passage à l'echelle (plein de tâches, plein de boutons, plein de
  boutons, plein de boutons\ldots{})
\end{itemize}


\subsection{« Et le bouton paramètre ? »}
\begin{itemize}
\item Résumer les actions possibles → Everything in one button
  limitation de la taille en changeant le texte (order/unorder)
\end{itemize}


\subsection{Ordonnancement}
\begin{itemize}
\item pas instinctif
\item pas vraiment une action majeure
  → suppressoin du "one click"
\end{itemize}


\subsection{Les actions majeures}
\begin{itemize}
\item check, expand, edit, close
  → one click
\end{itemize}


\subsection{Les actions mineures}
\begin{itemize}
\item add, templates, order, up\&down, sélection date
  → two clicks
\end{itemize}


\section{Placements}
\begin{itemize}
\item Note taken on \textit{[2013-02-21 Thu 10:54]} \\
  paper prototyping
\end{itemize}


\subsection{Des groupes}
\begin{itemize}
\item ajout, visualisation, résumé actions
\item description → car impartant
\item suppression → car critique
\end{itemize}


\subsection{Le bouton paramètre}
\begin{itemize}
\item votes
\end{itemize}


\subsection{La croix}
\begin{itemize}
\item votes
\end{itemize}


\subsection{Ajout d'une tâches principale}
\begin{itemize}
\item arrivée sur le logiciel → interface vide → un seul bouton
\end{itemize}


\section{Les couleurs}


\subsection{Les tâches}
\begin{itemize}
\item faite, à faire, en retard
\end{itemize}


\subsection{L'édition}
\begin{itemize}
\item le hover
\end{itemize}


\subsection{La croix}


\section{Le vocabulaire}

\begin{itemize}
\item ajouter tâche vs nouvelle tâche
\item tâche vs sous-tâche
\item template
\end{itemize}


\section{Les langues}
\begin{itemize}
\item menu édition
\item ajout d'une nouvelle langue rapide et facile
\end{itemize}


\section{Future works}
\begin{enumerate}
\item Drag\&drop
\item Shortcuts
\item Aide
\item Tout plier/déplier
\item Paramétrage des couleurs
\item Identité du logiciel (logo)
\item griser/supprimer le bouton expand
\end{enumerate}


\section{Bilan \& Conclusion}
\begin{itemize}
\item avantages expérimentations papier
\item prises en comptes retours utilisateurs
\end{itemize}


\end{document}
